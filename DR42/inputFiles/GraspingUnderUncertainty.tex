%%% Grasping under Uncertainty

\subsubsection{Grasping under Uncertainty}
\label{sec:GraspingUncertainty}

 This work extends that of \cite{bib:platt_csail_2011}, which offers a way to avoid the complexity of planning in a high dimensional belief space. It does this in two ways, i) by approximating the informational value of actions from a low-dimensional subspace of the belief state; and ii) by embedding that informational value into the physical space. Whereas Platt employed a 2DoF arm with a laser sensor, our extension allows reasoning about a 21 DoF arm-hand system with tactile observations. 

In our previous work~\cite{bib:zito_workshop_iros2012,bib:zito_iros_2013} we proposed a information gain re-planning strategy to improve the reliability of grasping under uncertainty. We achieved this by using a hierarchical sample-based path planner, here a Probabilistic Roadmap (PRM) planner, which explicitly encodes expected information gain (using a low-dimensional approximation of the belief state) along each of its trajectory segments. This extended the work in~\cite{bib:platt_csail_2011}. A particular contribution of our approach is refining the belief state for the object pose using an observation model for tactile sensing by a multi-finger hand that can palpate the object to be grasped. 

We have extended our previous work in four ways. First we now employ as the target grasp, one of the grasp candidates from the work on grasping of novel objects described elsewhere in this deliverable. Second we now enable re-planning from the configuration of the robot at the point where an unexpected contact occurs. This is instead of moving the hand away from the contact to a safe configuration. Third, we now plan dexterous grasping trajectories for non-convex object shapes by implementing an efficient collision detection method which reasons directly with sensed point clouds rather than derived representations of object shape such as mesh representations. Fourth, we now use an active compliant controller to ensure that continued contact(s) between the hand and the object minimise the risk of perturbing the system.

The updated method has been tested in simulation. Pose uncertainty is now caused by errors in matching a partial point cloud from one or two viewpoints to a previously sensed cloud from seven views. In each trial the simulated robot attempts to grasp the presented object using three different strategies. The first strategy represents a ``naive'' approach in which the robot tries to grasp the object in its estimated pose with a single attempt. The second strategy allows the robot to recover from unexpected contacts if the estimated pose is not correct, and a new grasp attempt is planned and executed until a termination condition is met. The third strategy is teh information gain strategy mentioned above. This works with a cost function that allows deviations from the shortest physical path in order to make tactile contacts that will reduce object pose uncertainty. Essentially this can be seen as warping the physical space using information gain to create a non-Euclidean distance metric
 for motion planning. The goal of the trials is to show that the information gain planner is superior, having a higher success rate, and requiring fewer re-planning iterations than the other methods before a grasp is achieved.
