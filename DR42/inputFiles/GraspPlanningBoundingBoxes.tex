%%% Grasp Planning for the Pisa/IIT Softhand via Bounding-Box Object
%%% Decomposition

\subsubsection{Grasp Planning for the Pisa/IIT Softhand via Bounding-Box Object Decomposition}
\label{sec:GraspPlanningBoundingBoxes}

In this section, we summarize the work performed to sharpen the approach of grasp planning and grasp acquisition for adaptive and compliant hands, such as the Pisa/IIT SoftHand. The ability we try embed in our grasp planner is that of selecting a successful grasp without the need of identifying \emph{a priori} an object, but only based on its composition of shape primitives with known associated grasps, i.e. resorting to \emph{part-based grasps}.

To this sake, we follow a mid-level solution, according to a purely top-down strategy, based on the Minimum Volume Bounding Box (MVBB) \emph{fit-and-split} algorithm to object decomposition that was originally proposed in~\cite{Huebner:ICRA:2008}. The method consists of iteratively building MVBBs of points resulting from splitting the point cloud of the object. The split procedure is performed in such a way that the sum of the two areas (proper box projections) resulting from the convex region of each set of points is minimized.

We adopt a modified version of the MVBB algorithm for object decomposition in~\cite{Huebner:ICRA:2008}, as it presents the following properties: (i) it is very efficient and can accomodate scattered 3D points delivered by arbitrary 3D sensors; (ii) the outcome constellation of boxes is quite insensitive to noise. However, with respect to~\cite{Huebner:IROS:2008} and~\cite{Geidenstam:RSS:2009}, where 2D grasp hypotheses are made and evaluated on point projections onto box faces and assume, from the outset, the use of rigid and fully-actuated robotic hands or grippers, in our work we propose a shift of prospective in the creation of grasp hypotheses as we employ a compliant and adaptive hand, the Pisa/IIT SoftHand. 

With a reduced burden to plan detailed finger placements, the box set representation of an object, also ease the mapping of complex actions to box and/or box distribution properties. For example, as also mentioned in~\cite{Huebner:IROS:2008}, in order to \emph{pick-up} an object and place it somewhere, it may intuitively be a good option to grasp the \emph{largest box}. Instead, in order to show the same object to a camera to gather more views, it may be preferable to grasp the object from the \emph{outermost box}, or from a box that allows a pincher grasp, so to minimize occlusions caused by hand parts.

Considering the adaptability of soft hands, we present a strategy to propose grasp postures to grasp a subset of the PaCMan kitchenware object database previously decomposed into MVBBs. The performances of the method are compared with those presented in~\cite{Bonilla:Humanoids:2015} and, through simulations performed with the MBS software ADAMS\texttrademark~\cite{ADAMS:ONLINE}, we show that with the strategy presented in this paper we increase the successful rate of grasps for the same objects.

More details on this new approach and on the achieved results can be found in the attached paper~\cite{Bonilla:IROS:2015} available at this~\href{./attachedPapers/GraspPlanningBoundingBoxes.pdf}{link}. 