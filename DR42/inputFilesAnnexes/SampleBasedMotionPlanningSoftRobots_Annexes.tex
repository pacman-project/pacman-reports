%%% Sample-Based Motion Planning for Soft Robot Manipulators
%%% Under Task Constraints

\item
\begin{list}{\quad}{}
\item \textbf{Paper title}: Sample-based motion planning for soft robot manipulators under task constraints
\item \textbf{Paper authors}: M. Bonilla, E. Farnioli, L. Pallottino, A. Bicchi
\item\textbf{Publication info}: \textit{IEEE Int. Conf. on Robotics and Automation (ICRA), 2015}
\item\textbf{Abstract}:
Random sampling-based methods for motion plan-
ning of constrained robot manipulators has been widely studied
in recent years. The main problem to deal with is the lack of an
explicit parametrization of the non linear submanifold in the
Configuration Space (CS), due to the constraints imposed by the
system. Most of the proposed planning methods use projections
to generate valid configurations of the system slowing the
planning process.
Recently, new robot mechanism includes compliance either
in the structure or in the controllers. In this kind of robot most
of the times the planned trajectories are not executed exactly
by the robots due to uncertainties in the environment. Indeed,
controller references are generated such that the constraint is
violated to indirectly generate forces during interactions.
In this paper we take advantage of the compliance of
the system to relax the geometric constraint imposed by the
task, mainly to avoid projections. The relaxed constraint is
then used in a state-of-the-art sub-optimal random sampling
based technique to generate any-time paths for constrained
robot manipulators. As a consequence of relaxation, contact
forces acting on the constraint change from configuration to
configuration during the planned path. Those forces can be
regulated using a proper controller that takes advantage of the
geometric decoupling of the subspaces describing constrained
rigid-body motions of the mechanism and the controllable
forces.
\item \textbf{Relation with the deliverable}: planning dual arm manipulation for soft robots.
\end{list}