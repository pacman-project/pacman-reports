%%% Grasp Planning for the Pisa/IIT Softhand via Bounding-Box Object
%%% Decomposition

%%% Grasping Planning Bounding Boxes

\item
\begin{list}{\quad}{}
\item \textbf{Paper title}: Grasp Planning for the Pisa/IIT Softhand via Bounding-Box Object Decomposition
\item \textbf{Paper authors}: M. Bonilla, D. Resasco, M. Gabiccini, A. Bicchi
\item\textbf{Publication info}: under review
\item\textbf{Abstract}:
  In this paper we present a method to plan grasps for Soft Hands. Considering that Soft Hands adapt to the shape of the object whatever its geometry is, we first approximate the object with bounding boxes. A set of hand poses are then proposed using geometric information of such bounding boxes. All hand postures are then used in a dynamic simulator of the PISA/IIT Soft Hand to evaluate if a proposed hand posture leads to a successful grasp or not.
We show through a set of simulations that the probability of success of the proposed hand poses is higher that with the existing methods.
At the end we show some experimental work using the PISA/IIT Soft Hand.

\item \textbf{Relation with the deliverable}: Part-based grasping under uncertainty for soft hands.
\end{list}