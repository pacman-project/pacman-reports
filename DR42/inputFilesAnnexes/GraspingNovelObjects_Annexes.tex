%%% Grasping Novel Objects

\item
\begin{list}{\quad}{}
\item \textbf{Paper title}: One shot learning and generation of dexterous grasps for novel objects
\item \textbf{Paper authors}: M. Kopicki, M. Adjible, R. Stolkin, A. Leonardis, R. Detry, J.L. Wyatt
\item\textbf{Publication info}: under review
\item\textbf{Abstract}: 
This paper presents a method for one-shot learning of dexterous grasps, and grasp generation for novel objects. A model
of each grasp type is learned from a single kinesthetic demonstration, and several types are taught. These models are used
to select and generate grasps for unfamiliar objects. Both the learning and generation stages use an incomplete point cloud
from a depth camera – no prior model of object shape is used. The learned model is a product of experts, in which experts
are of two types. The first is a contact model and is a density over the pose of a single hand link relative to the local object
surface. The second is the hand configuration model and is a density over the whole hand configuration. Grasp generation
for an unfamiliar object optimises the product of these two model types, generating thousands of grasp candidates in under
30 seconds. The method is robust to incomplete data at both training and testing stages. When several grasp types are
considered the method selects the highest likelihood grasp across all the types. In an experiment, the training set consisted
of five different grasps, and the test set of forty-five previously unseen objects. The success rate of the first choice grasp is
84.4\% or 77.7\% if seven views or a single view of the test object are taken, respectively.


\item \textbf{Relation with the deliverable}: A new method is proposed for learning grasps and generalize them to novel objects.
\end{list}