%%% Grasping With Soft Hands

\item
\begin{list}{\quad}{}
\item \textbf{Paper title}: Grasping with soft hands
\item \textbf{Paper authors}: M. Bonilla, E. Farnioli, C. Piazza, M. Catalano, G. Grioli, M. Garabini, M. Gabiccini, A. Bicchi
\item\textbf{Publication info}: \textit{IEEE-RAS Int. Conf. on Humanoid Robots (Humanoids) 2014}
\item\textbf{Abstract}: 
  Despite some prematurely optimistic claims, the ability of robots to
  grasp general objects in unstructured environments still remains far
  behind that of humans. This is not solely caused by differences in
  the mechanics of hands: indeed, we show that human use of a simple
  robot hand (the Pisa/IIT SoftHand) can afford capabilities that are
  comparable to natural grasping. It is through the observation of
  such human-directed robot hand operations that we realized how
  fundamental in everyday grasping and manipulation is the role of
  hand compliance, which is used to adapt to the shape of surrounding
  objects. Objects and environmental constraints are in turn used to
  functionally shape the hand, going beyond its nominal kinematic
  limits by exploiting structural softness.

  In this paper, we set out to study grasp planning for hands that are
  simple - in the sense of low number of actuated degrees of freedom
  (one for the Pisa/IIT SoftHand) - but are soft, i.e. continuously
  deformable in an infinity of possible shapes through interaction
  with objects. After general considerations on the change of paradigm
  in grasp planning that this setting brings about with respect to
  classical rigid multi-dof grasp planning, we present a procedure to
  extract grasp affordances for the Pisa/IIT SoftHand through
  physically accurate numerical simulations. The
  selected grasps are then successfully tested in an experimental
  scenario.
\item \textbf{Relation with the deliverable}: Grasping under uncertainty by exploiting the adaptive behavior encoded in the mechanical design.
\end{list}