\documentclass[a4paper,11pt,pdf]{pacmanreport}

\usepackage{helvet}
\usepackage{graphicx}
\graphicspath{{images/}{../shared_images/}}

% The following is used to make packages hyperref and cite work together
\makeatletter
\let\NAT@parse\undefined
\makeatother

\usepackage[bookmarks=true,hyperfootnotes=true]{hyperref}
\hypersetup{
			colorlinks=true,
			linkcolor=blue,
			anchorcolor=blue,
			citecolor=blue,
			urlcolor=blue,
            filecolor=blue,
			pdftitle={Deliverable 4.2}
}

%% ================================
%% PROJECT INFO

\project{}
\projectid{FP7-IST-60918}
\projectstart{1 March 2013}
\duration{36}

%% ================================
%% DELIVERABLE INFO

\title{Methodologies for incipient grasp evaluation}
\deliverableid{DR 4.2}
\author{H. Marino, M. Bonilla, D. Resasco, C.J. Rosales, M. Gabiccini}
\address{Universit\`{a} di Pisa, Pisa Italy}
\email{hamal.marino@centropiaggio.unipi.it}
\headertitle{Incipient grasp}
\headerauthor{H. Marino, M. Bonilla, D. Resasco, C.J. Rosales, M. Gabiccini}

\duedate{2015-02-28}
\submissiondate{2015-02-28}
\leadpartner{Universit\`{a} di Pisa}
\revision{draft}
\disseminationlevel{PU}


%% UNCOMMENT: to get the logo; if you've copied this file to a directory yearX/wpY/ then this should work
\reportlogo{pacmanlogo.png}


\begin{document}

\maketitle

\begin{abstract}
\noindent Here comes the abstract
\end{abstract}


\vspace{.2em}
\hrule

\footnotesize

\tableofcontents

\normalsize

\newpage

\section*{Executive Summary}

This report describes...

\section*{Role of incipient grasps evaluation in PaCMan}

What role do the tasks addressed in this report play in the larger context of PaCMan? How do these tasks, how does report, contribute to achieving the overall goals for PaCMan?

\section*{Contribution to the PaCMan scenario}

How do the results presented in this report contribute to the PaCMan scenarios and prototypes?


\newpage

\section{Tasks, objectives, results}

\subsection{Planned work}

What tasks was the report supposed to address? What objectives, results were these tasks to achieve?

\subsection{Actual work performed}

What does the report actually present? How have the tasks been addressed? To what extent have the intended objectives been achieved? Why, how -- or why not?

% \subsection{Relation to the state-of-the-art}
% How are the obtained results related to the state-of-the-art?
% This part is usually discussed in the corresponding subsection. Therefore, % a global 'Relation to the state-of-the-art' is unnecessary

%%%%%%%%%%%%%%%%%%%%%%%%%%%%%%%%%%%%%%%%%%%%%%%%%%%%%%%%%%%%%%%%%%%%%%%%%%%%%%
%%%%%%%%%%%%%%%%%%%%%%%%%%%%%%%%%%%%%%%%%%%%%%%%%%%%%%%%%%%%%%%%%%%%%%%%%%%%%%

% Work directly into the following .tex files

%%%%%%%%%%%%%%%%%%%%%%%%%%%%%%%%%%%%%%%%%%%%%%%%%%%%%%%%%%%%%%%%%%%%%%%%%%%%%%
%%%%%%%%%%%%%%%%%%%%%%%%%%%%%%%%%%%%%%%%%%%%%%%%%%%%%%%%%%%%%%%%%%%%%%%%%%%%%%

% To be prepared by: Marco Gabiccini
% Grasp Planning via a Bounding Box Object Decomposition
%%% Grasping With Soft Hands

\subsubsection{Grasping With Soft Hands}

In this section, we summarize the work performed to tackle the problem of grasp planning and grasp acquisition for hands that are simple --- in the sense of low number of actuated degrees of freedom (one degree of actuation for the Pisa/IIT SoftHands) --- but are soft, i.e. continuously deformable in an infinity of possible shapes through interactions with objects. This scenario presented us many different issues that could hardly be reconciled with the more classical approach to grasping --- to be performed with fully-actuated and sensorized robotic hands --- which is entailed in the consecutive temporal stages of grasp formation described Tasks 4.1, 4.2 and 4.3. Here, we face them all simultaneously, and we have to employ an adaptive underactuated hand to perform the task. This should explain why the material presented in this section is not exactly aligned to what the description of the Tasks presents. However, we believe that the work performed share with the description of the Tasks the same ultimate goals: robust grasping in uncertain scenarios.
  
 This research topic was started during the first year of the project --- it was already documented in DR 4.1 in the preliminary form of a technical report --- and has now reached a higher level of maturity, as testified by the accompanying paper~\cite{Bonilla:Humanoids:2015}.

During the past thirty years, the problem of autonomous grasping has been one of the most widely investigated. Several approaches have been proposed to define the optimal finger placement on the object, either based on geometric or force features, on grasp quality measures, specifically tailored to convex objects or generalized to non-convex ones. Perhaps because of the fragility of the mechanics of most robot hands, the multitude of the planning methods were thought for interactions between the hand and the objects that only occur at the fingertips, limiting contacts with other parts of the hand and avoiding contacts with the rest of the environment at all. This ``timid'' approach to manipulation generated by rigidity and fragility of the hand has been recently challenged by the introduction of adaptable, underactuated and/or soft hands. This allows to use these hands in a more ``daring'' way with the objects in the environment, using their full surface for enveloping grasps, and exploiting object and environmental constraints to functionally shape the hand, going beyond its nominal kinematic limits by exploiting structural softness.

The analysis of these possibilities constitute a new challenge for existing grasping algorithms: adaptation to totally of partially unknown scenes remains a difficult task, towards which only some approaches have been investigated so far. 

Here, we present a first approach to explore this novel kind of manipulation, based on an accurate simulation tool for the SoftHand, developed using the multi-body system software ADAMS. A batch simulation setup was created and used to perform the automatic creation of a database of grasp affordances for the Pisa/IIT SoftHand on the PaCMan database of kitchenware objects. 

More details on this new approach and on the achieved results can be found in the attached paper~\cite{Bonilla:Humanoids:2015} available at this~\href{./attachedPapers/GraspingWithSoftHands.pdf}{link}. 

%%%%%%%%%%%%%%%%%%%%%%%%%%%%%%%%%%%%%%%%%%%%%%%%%%%%%%%%%%%%%%%%%%%%%%%%%%%%%%

% To be prepared by: Manuel Bonilla
% Grasp Planning via a Bounding Box Object Decomposition
%%% Grasp Planning for the Pisa/IIT Softhand via Bounding-Box Object
%%% Decomposition

\subsubsection{Grasp Planning for the Pisa/IIT Softhand via Bounding-Box Object Decomposition}
\label{sec:GraspPlanningBoundingBoxes}

In this section, we summarize the work performed to sharpen the approach of grasp planning and grasp acquisition for adaptive and compliant hands, such as the Pisa/IIT SoftHand. The ability we try embed in our grasp planner is that of selecting a successful grasp without the need of identifying \emph{a priori} an object, but only based on its composition of shape primitives with known associated grasps, i.e. resorting to \emph{part-based grasps}.

To this sake, we follow a mid-level solution, according to a purely top-down strategy, based on the Minimum Volume Bounding Box (MVBB) \emph{fit-and-split} algorithm to object decomposition that was originally proposed in~\cite{Huebner:ICRA:2008}. The method consists of iteratively building MVBBs of points resulting from splitting the point cloud of the object. The split procedure is performed in such a way that the sum of the two areas (proper box projections) resulting from the convex region of each set of points is minimized.

We adopt a modified version of the MVBB algorithm for object decomposition in~\cite{Huebner:ICRA:2008}, as it presents the following properties: (i) it is very efficient and can accomodate scattered 3D points delivered by arbitrary 3D sensors; (ii) the outcome constellation of boxes is quite insensitive to noise. However, with respect to~\cite{Huebner:IROS:2008} and~\cite{Geidenstam:RSS:2009}, where 2D grasp hypotheses are made and evaluated on point projections onto box faces and assume, from the outset, the use of rigid and fully-actuated robotic hands or grippers, in our work we propose a shift of prospective in the creation of grasp hypotheses as we employ a compliant and adaptive hand, the Pisa/IIT SoftHand. 

With a reduced burden to plan detailed finger placements, the box set representation of an object, also ease the mapping of complex actions to box and/or box distribution properties. For example, as also mentioned in~\cite{Huebner:IROS:2008}, in order to \emph{pick-up} an object and place it somewhere, it may intuitively be a good option to grasp the \emph{largest box}. Instead, in order to show the same object to a camera to gather more views, it may be preferable to grasp the object from the \emph{outermost box}, or from a box that allows a pincher grasp, so to minimize occlusions caused by hand parts.

Considering the adaptability of soft hands, we present a strategy to propose grasp postures to grasp a subset of the PaCMan kitchenware object database previously decomposed into MVBBs. The performances of the method are compared with those presented in~\cite{Bonilla:Humanoids:2015} and, through simulations performed with the MBS software ADAMS\texttrademark~\cite{ADAMS:ONLINE}, we show that with the strategy presented in this paper we increase the successful rate of grasps for the same objects.

More details on this new approach and on the achieved results can be found in the attached paper~\cite{Bonilla:IROS:2015} available at this~\href{./attachedPapers/GraspPlanningBoundingBoxes.pdf}{link}. 

%%%%%%%%%%%%%%%%%%%%%%%%%%%%%%%%%%%%%%%%%%%%%%%%%%%%%%%%%%%%%%%%%%%%%%%%%%%%%%

% To be prepared by: Hamal Marino
% High-Level Planning for Dual Arm Goal-Oriented Tasks
%%% High-Level Planning for Dual Arm Goal-Oriented Tasks

\subsubsection{On the Problem of Moving Objects with Autonomous Robots: a Unifying High-Level Planning Approach}
\label{sec:HighLevelPlanning}

In order to successfully complete Task~5.3, which requires the object to be passed between the hands of the robot, we describe in this section our approach to solve high-level planning for this matter.

Our idea considers the possibility of having the object in a known position, which can be recognized using vision, and a higher-level agent (such as a user) decides a target configuration the object has to reach.

From this information, a semantic graph is constructed which has as nodes a grasp id and a workspace id, and as arcs all possible known transitions between nodes.

Once a minimum cost path has been found, it is translated back into Cartesian information for collision-free motion planning, grasp commands, and all low-level requirements to execute the plan successfully.

While there are many previous works on combining high-level semantic reasoning and low-level cartesian path planning (see e.g. \cite{karlsson2012combining, leidner2012things, leidner2013hybrid}) which mostly take advantage of object-specific reasoning introduced in \cite{levison1996connecting}, our main contribution is to include the possibility of passing an object between two end-effectors.

Specifically, considering end-effectors with different properties, the table (and possibly other non-movable surfaces) has its own set of grasps for an object, and is treated exactly like a hand when generating arcs in the graph, apart from the fact that it cannot be moved and, thus, there will be no arc connecting the same ``table grasp'' in adjacent workspaces (while there could be one if the grasp is performed via a hand of the robot).

Thus, considering that previous approaches mostly rely on a table in order to move an object from one arm workspace to another, we can simply classify the \emph{primitives} which allow a transition from a grasp/workspace pair into another in three categories:
\begin{enumerate}
	\item \emph{pick} with an end-effector from a fixed surface
	\item \emph{move} the end-effector
	\item \emph{place} with an end-effector onto a fixed surface
\end{enumerate}

Our approach uses instead a more general action of ``grasp transfer'', which can be specialized in:
\begin{enumerate}
	\item moving actions of an end-effector among its reachable workspaces
	\item pick and place actions if one end-effector involved is non-movable (such as a table)
	\item a grasp-ungrasp sequence if both end-effectors involved are movable
\end{enumerate}
This list is still under development, and we will try to generalize it even more in the future with other, more interesting \emph{primitives} which could possibly exploit dynamics, friction, gravity, and so on.

A more detailed view on this method and on the achieved results can be seen in Sec.~\ref{ann:highLevelPlanning}. % available at this~\href{./attachedPapers/HighLevelPlanningBimanualObjectPassing.pdf}{link}.


%%%%%%%%%%%%%%%%%%%%%%%%%%%%%%%%%%%%%%%%%%%%%%%%%%%%%%%%%%%%%%%%%%%%%%%%%%%%%%

% To be prepared by: Marco Gabiccini
% Computational Framework for Environment-Aware Robotic Manipulation 
% Planning
%%% A Computational Framework for Environment-Aware Robotic Manipulation 
%%% Planning

\subsubsection{A Computational Framework for Environment-Aware Robotic Manipulation Planning}

Moving beyond with respect to what we committed to investigate in Tasks 4.2 and 4.3, we introduce, in this section, a computational framework for direct trajectory optimization of general manipulation systems without \textsl{a-priori} specified contact sequences, possibly exploiting \emph{environmental constraints} as a tool to accomplish a task.
 
 Originally, we planned to approach the problem of robust grasping by dividing it into three main stages: (i) move from hand pre-shape to first object contact (Task 4.1), (ii) perform an incipient grasp and assess the first-order properties of the surface of the object, possibly changing candidate contact locations so that the quality of the incipient grasp is maximized (Task 4.2), and (iii) perform the actual grasp acquisition by optimizing the distribution of the applied contact forces (Task 4.3).
 
 In this section, we report on our efforts to devise a framework to be employed not only for grasp planning, but also for manipulation planning, that should be able to merge all three previous phases in a systematic and coherent way. The user should be focused on providing high-level objectives, e.g. ``move object A from pose 1 to pose 2 in a given amount of time'', and the framework should be able to provide a manipulation plan that should take care of all the rest, e.g. should specify low-level actions and their correct sequence for the manipulation system at hand (single-arm configuration, dual-arm configuration),  defining the whole contact sequence (where and when to make and to break contacts), should provide trajectories consistent with the dynamics of the manipulation system, unilateral contact and friction constraints and actuation limits. Then, if the task may be more efficiently, and/or more robustly, performed with the aid of constraints provided by the environment, the proposed approach should devise manipulation strategies that cleverly and opportunistically exploit environmental constraints.
 
 To this sake, two approaches are presented to model the dynamics of systems with intermittent contacts: the first one, in which continuous contact reaction forces are generated  by nonlinear virtual springs, is convenient to tackle scenarios where we try to avoid \emph{sliding} contacts; the second one, which is based on a velocity-based time stepping scheme, is suitable in scenarios where \emph{sliding} interaction primitives may lead to convenient interactions among the hand, the manipulandum and the environment.

In both cases, beside system's state and applied torques, object and environment contact forces are included among the free optimization variables, and they are rendered consistent via suitably devised sets of \emph{complementarity} conditions. To maximize computational efficiency, sparsity patterns in the linear algebra expressions generated during the solution of the optimization problem are exploited, and Algorithmic Differentiation (also known as Automatic Differentiation) is leveraged to calculate derivatives. 

The approach is evaluated in three simulated planar manipulation tasks: (i) moving a circular object from an initial pose to a final pose in the workspace with two independent fingers, (ii) rotating a capsule-shaped object with an underactuated two-fingered gripper, and (iii) rotating a circular object in hand with three independent fingers. Tasks (i) and (ii) show that our algorithm quickly converges to locally optimal solutions that opportunistically exploit environmental constraints. Task (iii) demonstrates that even dexterous fingertip gaits can be obtained as a special solution in the very same framework.

More details on this new approach and on the achieved results can be found in the attached paper~\cite{Gabiccini:RSS:2015} available at this~\href{./attachedPapers/ComputationalFrameworkEnvAwareRobManipPlanning.pdf}{link}. 

%%%%%%%%%%%%%%%%%%%%%%%%%%%%%%%%%%%%%%%%%%%%%%%%%%%%%%%%%%%%%%%%%%%%%%%%%%%%%%

\section{Annexes}

Which papers / articles are included in the report? Mention titles, authors, publication info; abstract; and a one-liner relating the publication back to the discussion on actual work performed.

\bibliographystyle{IEEEtran}
\bibliography{./bibliography/abbreviations,./bibliography/marco}


\end{document}
