%%% Grasping With Soft Hands

\subsubsection{Grasping With Soft Hands}

In this section, we summarize the work performed to tackle the problem of grasp planning and grasp acquisition for hands that are simple --- in the sense of low number of actuated degrees of freedom (one degree of actuation for the Pisa/IIT SoftHands) --- but are soft, i.e. continuously deformable in an infinity of possible shapes through interactions with objects. This scenario presented us many different issues that could hardly be reconciled with the more classical approach to grasping --- to be performed with fully-actuated and sensorized robotic hands --- which is entailed in the consecutive temporal stages of grasp formation described Tasks 4.1, 4.2 and 4.3. Here, we face them all simultaneously, and we have to employ an adaptive underactuated hand to perform the task. This should explain why the material presented in this section is not exactly aligned to what the description of the Tasks presents. However, we believe that the work performed share with the description of the Tasks the same ultimate goals: robust grasping in uncertain scenarios.
  
 This research topic was started during the first year of the project --- it was already documented in DR 4.1 in the preliminary form of a technical report --- and has now reached a higher level of maturity, as testified by the accompanying paper~\cite{Bonilla:Humanoids:2015}.

During the past thirty years, the problem of autonomous grasping has been one of the most widely investigated. Several approaches have been proposed to define the optimal finger placement on the object, either based on geometric or force features, on grasp quality measures, specifically tailored to convex objects or generalized to non-convex ones. Perhaps because of the fragility of the mechanics of most robot hands, the multitude of the planning methods were thought for interactions between the hand and the objects that only occur at the fingertips, limiting contacts with other parts of the hand and avoiding contacts with the rest of the environment at all. This ``timid'' approach to manipulation generated by rigidity and fragility of the hand has been recently challenged by the introduction of adaptable, underactuated and/or soft hands. This allows to use these hands in a more ``daring'' way with the objects in the environment, using their full surface for enveloping grasps, and exploiting object and environmental constraints to functionally shape the hand, going beyond its nominal kinematic limits by exploiting structural softness.

The analysis of these possibilities constitute a new challenge for existing grasping algorithms: adaptation to totally of partially unknown scenes remains a difficult task, towards which only some approaches have been investigated so far. 

Here, we present a first approach to explore this novel kind of manipulation, based on an accurate simulation tool for the SoftHand, developed using the multi-body system software ADAMS. A batch simulation setup was created and used to perform the automatic creation of a database of grasp affordances for the Pisa/IIT SoftHand on the PaCMan database of kitchenware objects. 

More details on this new approach and on the achieved results can be found in the attached paper~\cite{Bonilla:Humanoids:2015} available at this~\href{./attachedPapers/GraspingWithSoftHands.pdf}{link}. 