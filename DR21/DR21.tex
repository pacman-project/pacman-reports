\documentclass[a4paper,11pt,pdf]{pacmanreport}

%%=== Aditional packages
%\usepackage{natbib}
%\setcitestyle{round}
% The following is used to make packages hyperref and cite work together
\makeatletter
\let\NAT@parse\undefined
\makeatother
\usepackage[bookmarks=true,hyperfootnotes=true,colorlinks=true,linkcolor=blue,anchorcolor=blue,citecolor=blue,urlcolor=blue,filecolor=blue]{hyperref}

%%=== Local definitions
\graphicspath{{images/}{../shared_images/}}

%% ================================
%% PROJECT INFO 

\project{}
\projectid{FP7-IST-60918}
\projectstart{1 March 2013}
\duration{36}

%% ================================
%% DELIVERABLE INFO 

\title{Multi-modal compositional hierarchies}
\deliverableid{DR 2.1}
\author{Mirela Popa, Safoura Rezapour-Lakani, Alexander Rietzler, Justus Piater}
\address{Institute of Computer Science, University of Innsbruck}
\email{Mirela.Popa@uibk.ac.at}
\headertitle{Multi-modal compositional hierarchies}
\headerauthor{M.~Popa et al}

\duedate{2015-02-28}
\submissiondate{2015-02-28}
\leadpartner{UIBK}
\revision{final}
\disseminationlevel{PU}


%% UNCOMMENT: to get the logo; if you've copied this file to a directory yearX/wpY/ then this should work
\reportlogo{pacmanlogo}


\begin{document}

\maketitle

\begin{abstract}
\noindent This deliverable describes the intermediate results of Tasks 2.1 and 2.2 as specified in the DoW. In this report the representations of shape and the inference methods developed are described along with the way in which grasps are attached to the object representations.
\end{abstract}

\vspace{.2em}
\hrule

\footnotesize

\tableofcontents

\normalsize

\newpage

\section*{Executive Summary}

This report presents work carried out in WP2 on multi-modal compositional hierarchies of object categories, using both visual and non-visual features. The work addresses the intermediate results of Tasks 2.1, 2.2 and 2.3. We describe an approach for integrating non-visual features into the compositional hierarchy, as defined in Task 2.1.
%, which is presented in detail in Annex \ref{ann:techReport}. 

Next, we present the association of graspability to object parts formed using a hierarchical representation, according to Task 2.2. In addition, we propose a method for structuring compositional object models into semantically meaningful parts, guided by their graspability.
%, which is described in a submitted conference paper included in Annex \ref{ann:icra}.

\section*{Role of the Multi-modal compositional hierarchies in PaCMan}

In WP2, we focus on integrating non-visual features into the multi-modal compositional hierarchy. Furthermore, we propose forming semantically meaningful parts based on their functionality. Both these aspects contribute to a better understanding of the environment and  will be used to refine object hypothesis, decompose an object into graspable parts, as well as reasoning under incomplete observations, either visual or non-visual.

\section*{Contribution to the PaCMan scenario}

The proposed algorithms will be used to process visual and non-visual information and to support reasoning under uncertainty in the dishwasher-scenarios addressed in WP5.

\newpage

\section{Tasks, objectives, results}

\subsection{Planned work}

DR 2.1 addresses multi-modal compositional hierarchies for object representation using both visual and non-visual information. Planned work mainly concerns Task 2.1 regarding integration of haptic features into the compositional model, Task 2.2 regarding integration of graspability into the compositional 3D model, Task 2.3 regarding structuring the 3D compositional model according to graspability, and Task 2.4 regarding the creation of scene models from sensor data.

The objective of Task 2.1 consists of defining a suitable parametrization of haptic features. Next, it addresses the development of methods for incremental acquisition of haptic information and its incorporation into the multi-modal representation. 

Task 2.2 aims at associating grasps to object parts learned using a compositional approach.

In Task 2.3, algorithms for structuring the 3D compositional models based on graspability are designed and implemented.

Task 2.4 addresses the detailed description of a scene, using all types of information provided by the multi-modal representation, such
as object models described by their pose, as well as any visual, haptic and graspability properties associated with the objects.

\subsection{Actual work performed}

In this section, the main achievements related to the topic of this deliverable are briefly described. 
%For detailed descriptions of the work performed, the reader is referred to the papers and technical reports in Annex \ref{ann} of this deliverable.
We choose to introduce first Task 2.3, as it describes the developed compositional model which is used in Tasks 2.1, 2.2 and 2.4, facilitating a better understanding of the performed work.

\subsubsection{Task 2.3 Structuring 3D compositional models according to grasp affordances}

We designed and implemented a method for structuring the 3D compositional models according to object part graspability.

In order to associate graspability to an object part and to integrate it with tactile information, this work uses a hierarchical method model that explicitly seeks to construct semantically-meaningful parts, rather than the bottom-up approach developed in Task 1.2 in WP1. We are now also investigating the attachment of grasps to 3D parts developed using the bottom-up approach of WP1.

Object are represented in a compositional hierarchy where their parts are described by the relationship between sub-parts, which are subsequently represented based on the relationships between small adjacent patches. 

We learn our model based on view-based RGB-D pointcloud data. We have developed a bottom-up compositional model which starts from the object points and reaches to the object category level.

Initially, the points form low-level patches. These patches construct the lowest level of our compositional model. In order to capture the different patch types in our dataset, we extract features from them, which quantize the distribution of normals inside the patch and
perform clustering to form a patch codebook.

The next level in our hierarchical model is represented by object parts, which are composed of low-level patches. We describe each object part as a histogram of the patch types which compose it. This histogram representation constructs our visual features. Furthermore, we perform clustering on our object parts to obtain different part types and form a part codebook.

Finally, the object level is composed of a set of parts in a specific
configuration. The notion of an object part is difficult to define,
and is often in the eye of the beholder.  Our objective is to form
parts that are meaningful in robotic interaction.  Thus, parts are
formed from object regions observed to be involved in robotic grasps,
which is a particular novelty of our approach. An object is grasped in
different ways, depending on its intended functionality.  Information
gathered during grasping experiments is exploited in our framework for
forming semantically meaningful parts.

Furthermore, from the conceptual view of our hierarchical framework, our objective is to collect two types of information. First, we are interested in computing the probabilistic occurrence of a certain part in an object class. Secondly, we aim at learning the relations between the parts.

For collecting the first type of information, we segment an object
into its constituent parts, and assign each part to an entry in the
part codebook. From the obtained part types, we collect statistics of
occurrence of each part type in a certain object class.

The second type of information, relations between parts, cannot be
obtained at the part level. The reason is that we represented a part
as a histogram of patch types. This representation only quantizes the
occurrences of patch types, but it does not contain any information
about the spatial distribution of patches inside a part.

In order to overcome this problem, we make use of the compositional
nature of our model, by further analyzing the patches which form a
part. However, we do not use all the patches inside a part, but only
those patches adjacent to the border between two adjacent parts. Our
assumption relies on the fact that object instances which belong to
one class have the same connection between the parts. As an example,
mugs can have different handles and different bodies but the bodies
and handles are always connected with a hyperbolic surface.

Next, we recognize these boundary patches given our patch codebook,
and we collect statistics of their co-occurrence for a certain object
class.

An illustration of the described approach is depicted in Figure \ref{fig:rel}.

\begin{figure}[h!]
\begin{center}
\includegraphics[width=0.7\textwidth]{rel.pdf}
\end{center}
\caption{A schematic diagram of the developed compositional model, for one object class $o$. For each object, the observed patches $\mathbf{Q}=\{\mathbf{q_1},\ldots,\mathbf{q_4}\}$ are matched to the patch codebook $Q=\{q_1,\ldots,q_{N_{Q}}\}$. An object part is represented as a histogram of patch types. An object from class $o$ consists of a set of parts $\mathbf{R}=\{\mathbf{r_1},\mathbf{r_2},\mathbf{r_3}\}$ which are matched to the part codebook $R=\{r_1,\ldots,r_{N_{R}}\}$. For enabling object recognition, we learn the probabilities $p(r_1,\ldots,r_{N_{R}} \vert o)$ as well as parts connectivity based on the boundary patches $p(q_{1},q_{2} \vert o)$.} 
\label{fig:rel}
\end{figure}

The novelty of our approach is two-fold. First, the proposed algorithm forms semantically meaningful object parts in a hierarchical manner, by exploiting the object functionality, and more specifically in our case the object graspability. Next, it provides a generalization mechanism for recognizing novel object instances, by exploiting the relations between adjacent parts, where the composition at each level of the hierarchy is learned from grasping experience.

We evaluated our method on our collected Ikea Object Dataset and on the RGB-D Washington dataset \cite{rgbd-dataset}. Our approach achieved promising results, outperforming competing methods \cite{rel7}.

%More details can be found in Annexes \ref{ann:icra} and \ref{ann:cvpr}.
More details can be found in Sec.~\ref{ann:icra} and~\ref{ann:cvpr}. %, available at~\href{./attachedPapers/rezapouretal-cvpr2015.pdf}{this link} and~\href{./attachedPapers/rezapouretal-icra2015.pdf}{this link}, respectively.


\subsubsection{Task 2.1 Integrating haptics into 3D compositional models}

To achieve the objectives of Task 2.1, we have taken several steps. First, we implemented a framework for acquiring tactile information simultaneously with visual data, using our robotic framework. 

Next, we recorded a dataset of Ikea objects, consisting of RGB and depth data, grasping information and tactile readings corresponding to the grasped patches of the object. In these experiments we used a parametrization suitable for our Schunk hand. Furthermore, using our developed compositional framework presented in Sec.~\ref{ann:cvpr}, we decompose an object into its constituent parts and associate the corresponding tactile features. The integration of visual and non-visual features was done in a probabilistic manner, as described in Sec.~\ref{ann:techReport}. One application of the proposed representation is object class recognition under uncertainty.

The actual steps required for extracting tactile features and fusing them with visual features are:
\begin{enumerate}

\item Extract features from the tactile readings.
\item Associate the extracted features to the part level of the hierarchy.
\item Formulate in a probabilistic manner the binding of haptic and visual features.
\item Propagate the visual and non-visual information in the
  hierarchy, in a bottom-up manner, to allow inference of one type of
  information in the absence of the other.
\end{enumerate}

\comment{[MP] New details will be added under this section, based on the input received from Pisa and from one of our students.}

\comment{[CR] @Federico, please, smooth the following into this section, perhaps in a technical report annexed format would be better, otherwise you need to downgrade the sectioning}
%%%%%%%%%%%% To be added by Federico (UNIPI): a proper introduction inside the report, a relation to what is already written.
%%%%%% FOR NOW IM JUST PUTTING IT HERE, AT THE END, UNTIL A MORE SUITABLE PLACE IS FOUND
%%% Object Grasping Dataset 

\subsubsection{Object Grasping Dataset}
\label{sec:ObjectGraspingDataset}

In this section we present an object grasping dataset, collected at the University of Pisa, to be used within the PaCMan project as a tool to perform high level planning and grasping.
The goal of this dataset is to provide a grasp database for each object so that the relative pose between object and hand is always known during the grasp. This information can be later 
exploited to reproduce the grasp autonomously. 
The dataset is composed by sub-datasets, each one containing a series of grasps performed on a kitchen environment object. The grasps are performed by a human operator, wearing
a motorized handle on his right arm, which supports and operates the Pisa/IIT SoftHand, the operator drives the SoftHand to grasp the object put on a table. 
\begin{figure}[!tb]
  \centering
  \includegraphics[width=0.955\textwidth]{GRP_subdb1.png}
  \caption{Pictures taken during the recording of the jug sub-dataset. The operator drives the SoftHand towards the object, then grasps and lifts it, while sensor data is being recorded.}
  \label{fig:grasp:subdb1}
\end{figure}
Figure~\ref{fig:grasp:subdb1} show photographs taken during the acquisition of a sub-dataset.
In each sub-dataset several different grasp types were performed, according to the object shape. A goal we kept in mind was to mimic human behaviour in grasping everyday objects, for instance a cup was grasped by the handle, 
by the top without putting fingers inside it or by the sides and so on\ldots
Each grasp recorded is composed by a pre-grasp phase and an actual grasp phase. In the first the operator moves the SoftHand towards the object and starts closing over the designated spot and in the latter the SoftHand 
closes on the object and it is lifted up for a few seconds, then the object is put back on the table and the record stops. The user is able to distinguish between these two phases by reading sensor data: for instance joint positions
of each SoftHand finger is being recorded, so one can notice when the hand starts closing on the object or when it is just moving towards it.
The dataset is then populated by 8-10 seconds long records, each classified in sub-datasets according to the object that is grasped. On each record the user has constant access to relative and/or absolute hand posture, object pose,
hand joints positions and point clouds of the whole scene. In fact the philosophy of the dataset was to collect as much data as possible during the grasp and then give the user the flexibility to chose which data to use for his application.
The remainder of the section describes which hardware was used during records and how it was configured, which software was used and finally a description and usage of data.

\paragraph{(a) Hardware setup used during dataset recording:}
%description of various hardware used
The whole system used to capture grasp recordings is composed by the following subsystems:
\begin{itemize}
  \item PhaseSpace Impulse Motion Tracking System.
  \item Pisa/IIT SoftHand.
  \item Handle for SoftHand.
  \item Flexi Force glove for SoftHand.
  \item Asus Xtion Pro Live.
\end{itemize}

The PhaseSpace Impulse system captures real-time motion by using cameras and LEDs.
The cameras detect the positions of the LEDs, which can be identified via an ID, and transmit the position of each LED in real-time at a frequency of 480Hz.
The main use of this system is to track the position of the SoftHand and the object to be grasped, so that the user has access to these data. 
To accomplish this feature two star-like prints were created to hold five LEDs each, one star was fixed to the SoftHand, the other on the object to be grasped.
Once calibrated the system identifies the two stars and attach a local reference system to them, which they give, respectively, the pose of the SoftHand and the object in space. %maybe explain better
In Figure~\ref{fig:grasp:star} pictures of the stars attached to the object and the SoftHand are visible.
\begin{figure}[b!]
  \centering
  \includegraphics[width=0.95\textwidth]{GRP_star}
  \caption{The star with five LEDs used to track the pose of objects (left). The star used to track the SoftHand (right). The five unique LED IDs and their relative position, identify a local reference system for each star, which are then tracked by the PhaseSpace system.}
  \label{fig:grasp:star}
\end{figure}

The Pisa/IIT SoftHand was dressed with a special glove, called Flexi Force Glove, also designed at the University of Pisa.
Glove readings give a measure of finger bending and consequently joint angle values are estimated to be one third of the finger bending. This is a good approximation considering 
the SoftHand synergies and the type of application. Figure~\ref{fig:grasp:hand_glove} shows the SoftHand with and without the glove.
\begin{figure}[tbh!]
  \centering
  \includegraphics[width=0.95\textwidth]{GRP_hand_glove}
  \caption{The Pisa/IIT SoftHand without the FlexiForce Glove (left) and another one with it (right). Note the sensors alongside the hand fingers that measure the angular displacements.}
  \label{fig:grasp:hand_glove}
\end{figure}

Additionally, to ease the operator in manoeuvring the SoftHand a special support handle was created and used. This device can be attached to the operator's arm so that he can efficiently move and operate the SoftHand, similarly as he would with his own hand.
The handle has a lever to open and close the SoftHand with ease, a photograph of this device is visible in Figure~\ref{fig:grasp:handle}.
\begin{figure}[b!]
  \centering
  \includegraphics[width=0.95\textwidth]{GRP_handle}
  \caption{The handle for SoftHand without (left) and with (right) the operator's arm. The lever to close the hand is located where the operator's hand should rest.}
  \label{fig:grasp:handle}
\end{figure}

Finally the Asus Xtion Pro Live RGB-D sensor was used to record point clouds and images of the scene while the grasp was recorded. This not only provides the user with more visual feedback, but it also gives more data that
could be post processed to analyze the grasp in a perceptual sense.
\paragraph{(b) Software setup used during dataset recording:}
%1mention we used ROS to sync various hardware
%2talk about calibration steps, and pose estimation paper
%3link and mention github repository where the software is stored
The software employed was developed under the Robot Operating System (ROS) framework. ROS is an open-source set of software and tools widely used in robotics community,
that provides structured communications between heterogeneous hardware designed for robotics application in general.
For our scenario we needed a common platform to communicate with all the various hardware we used and record meaningful data.
The software is available at \href{https://github.com/CentroEPiaggio/unipi-grasp-datasets/tree/master/scenario1}{our GitHub repository} along with a description of the various packages used and the procedure to replicate the grasp
recordings and playing them back. %put url in bib and url pacman project url
As a brief summary the software employs a communication framework like in Figure.~\ref{fig:grasp:software}, %add pic from tf or draw one
where all devices are configured and calibrated together. 
Calibration steps for the experiment are also described in detail on the project website %bib link
and covers:
\begin{itemize}
  \item Calibration between PhaseSpace system and Asus Xtion.
  \item Calibration between first star and SoftHand.
  \item Calibration between second star and the object to be grasped.
\end{itemize}

The first one has to performed just once at the initial setup phase of the experiment, the second was done by means of geometric transformations: knowing the SoftHand and star cad models and where the latter was fixed on the hand.
The last calibration instead is more involving, since it has to be performed every time a new object is selected for grasping. This calibration is performed by knowing the pose of the second star with respect to the PhaseSpace and the
pose of the object with respect to the Asus camera, now since the two systems are calibrated with each other (from first calibration), the pose of the object with respect to the star is obtained by concatenating known transformations.
The last calibration remains valid until another object is selected or the star is moved.
To obtain the pose of the object with respect to the camera we created a procedure to achieve pose estimation with rgb-d vision.
%brief description of the procedure


\paragraph{(c) Data collection and description:}
%explain exactly what is recorded and bagfiles
%data format: msgs, tfs, pointclouds
\paragraph{(d) Data usage:}
%talk about how to playback (link)
%talk about what we could extract from recordings and possible uses (grasp planning)
\paragraph{(e) Data access methods:}
%where the data is stored and how a user can access it





%%%%%%%%%%%%%%%%%%%%%%%%%%%%%%%%%%%%%%%%%%%%%%

\subsubsection{Task 2.2 Integrating graspability into 3D compositional models}

We associated grasps to object parts, obtained using our proposed compositional framework. This approach facilitates grasping unknown objects, which are composed of parts similar to those learned by our hierarchical model. To facilitate this task, we have designed and implemented a compositional framework which is described in detailed in Sec.~\ref{ann:cvpr}. The description of the grasp representation and of the experimental setup is described in Sec.~\ref{ann:techReportAlex}.%, available at~\href{./attachedPapers/rietzeler-techReport.pdf}{this link}.

The steps performed to associate grasps to object parts are:
\begin{enumerate}
\item Record a dataset of grasped object parts, by capturing object relative wrist pose of the gripper, tactile signature and kinesthetic signature (gripper joints).
\item Detect and localize object parts using our proposed compositional model introduced in Task 2.3.
\item Form a set of 3D shape templates according to different grasp types (pinch, spherical, and parallel).
\item Given a scene, align the set of 3D shape templates to the cropped point-cloud sections, provided by the part recognition module and chose the best one according to an optimization procedure.
\item Evaluate the association of grasps to object parts, by performing robotic experiments on a test set containing objects partially similar to the ones in the training set.
\end{enumerate}

The grasp template alignment procedure is based on our pose estimation algorithm~\cite{detry2010ac}, where the 6D pose dependent cross correlation between the shape distribution of the scene and a template is computed. 

Next, in order to find a single best grasp for each template, we have developed a grasp inference algorithm which searches over a large number of hypothesis, while performing collision checking with the environment and checking inverse kinematics reachability of the gripper wrist pose hypothesis.

\textbf{We will add results of the grasp association evaluation.}

\subsubsection{Task 2.4 3D compositional models of the work space}

Reasoning about a scene can be very useful for action planning or grasp planning in the presence of clutter. Visual information plays the most important role, while additional information such as object graspability and non-visual features such as haptics, can be also useful for obtaining an improved and refined representation of the work space.

Therefore, we have analyzed first the performance of our developed 3D compositional model mainly on two aspects: part recognition for novel objects and object classification and localization. A detailed description of the achieved results is described also in Sec.~\ref{ann:cvpr}, while we present next a brief discussion of our main contributions.

Due to the compositionality nature of our object representation, we were able to obtain a very high recognition rate on novel object instances, composed of parts different from the ones in the training set. The generalization property of our model is due to the fact that we learn the relations between adjacent parts in an object and therefore our method is not fully dependant on the structure of a part. Furthermore, our approach outperformed state-of-the-art methods \cite{vfh}, which achieved poor results for novel object instances.

Next, we enriched the representation of a scene, by associating grasp templates to detected object parts, as described in Task 2.2.

Finally, as future work we plan to analyze the stability of a grasp using haptic information which can indicate to what extent the test grasp configuration is similar to training data.

\subsection{Relation to the state-of-the-art}

The main contribution of the results achieved for Task 2.3 lies in a successful design and implementation of a learnt compositional hierarchy. The main difference from other works \cite{rel2},\cite{comp2} introducing hierarchical compositional 3D object models, consists in the generalization and scale invariant properties of our representation. 

Furthermore, by exploiting the relations between object parts, we were able to recognize novel object instances having a similar structure, while the local appearance of a part was different from the learned examples. Our method achieved an accuracy of 85\% for the recognition of novel object instances, outperforming state-of-the-art methods \cite{vfh}. 

Moreover, our proposed method performed better at decomposing an object into parts, compared with \cite{rel7}, where the notion of object convexity and concavity was exploited for segmenting an object into parts.

In the state-of-the-art methods for part-based object recognition \cite{part2}, \cite{part1}, \cite{part3}, parts are manually labeled in the training examples, while the novelty of our approach consists in extracting the region corresponding to a part, from grasping experiments. An object is grasped in different ways, depending on its intended functionality, information which is exploited in our framework for forming semantically meaningful parts.

\newpage

\bibliographystyle{IEEEtran}
\bibliography{../shared_bibliography/abbreviations,./bibliography/DR21}

\newpage 

\appendix 

\section{Annexes}

%Which papers / articles are included in the report? Mention titles, authors, publication info; abstract; and a one-liner relating the publication back to the discussion on actual work performed. 

\subsection{Article: Learning 3D Object Parts through Grasping} \label{ann:icra}
\begin{description}
	\item[Authors] Safoura Rezapour Lakani, Mirela Popa, Antonio J. Rodriguez-Sanchez and Justus Piater.
	\item[Info] Submitted to the 2015 IEEE Computer Vision and Pattern Recognition. \comment{[CR] @Mirela @Safoura shuoldn't this be ICRA? If so, shouln't you have the decision alreay?}
	\item[Abstract] Most objects are composed of parts which have a semantic meaning. A handle can have many different shapes and be present in quite different objects, but there is only one semantic meaning to a handle, which is ``a part that is designed especially to be grasped by the hand''. We introduce here a novel 3D algorithm for the decomposition of objects into their semantically meaningful parts. These meaningful parts are learned from experiments where a robot grasps different objects. Object are represented in a compositional graph hierarchy where their parts are represented as the relationship between sub-parts, which are subsequently represented based on the relationships between small adjacent regions. Unlike other 2D compositional approaches, the importance of our 3D oriented method relies on the relationship between adjacent regions and sub-parts and where the composition at each level of the hierarchy is learned from grasping experience. We evaluate the validity of our compositional approach in three different scenarios, where we test the importance of the extracted relationships, its robustness to scale and more importantly, the recognition of parts from previously unseen objects.
	\item[Relation with the deliverable] The paper addresses Task 2.3 by introducing a method for learning objects parts in a bottom-up way, from robotic experiments of grasping objects.
    \item[Attachment] (following pages until next annex)
\end{description}
\includepdf[pages=-]{./attachedPapers/rezapouretal-icra2015.pdf}

\subsection{Article: Learning Part-Based 3D Compositional Object Representations} \label{ann:cvpr} 
\begin{description}
	\item[Authors] Safoura Rezapour Lakani, Antonio J. Rodriguez-Sanchez and Justus Piater.
	\item[Info] Submitted to the 2015 IEEE Computer Vision and Pattern Recognition.
	\item[Abstract] This paper presents a novel approach to parts-based object representation from depth images. We propose a bottom-up compositional model for representing object classes. Our model uses a probabilistic approach to build object representations starting from small patches that are successively built up into regions, and finally meaningful (possibly graspable) parts . Parts represented this way are the main representatives of the identity of an object class. We have evaluated our method at parts recognition and object categorization, outperforming competing methods.
	\item[Relation with the deliverable] The paper presents a probabilistic model for learning object representations in a hierarchical manner. This representation is useful for all the tasks, as it enables forming objects by graspability (Task 2.3), it allows integration of non-visual features (Task 2.1), and it can be used for a refined scene description (Task 2.4).
    \item[Attachment] (following pages until next annex).
\end{description}
\includepdf[pages=-]{./attachedPapers/rezapouretal-cvpr2015.pdf}

\subsection{Technical Report: Multi-Modal Compositional Object Representation} \label{ann:techReport}
\begin{description}
\item[Authors] Safoura Rezapour Lakani and Justus Piater.
\item[Info] Internal technical report.
\item[Abstract] The objective of this technical report is to introduce an object representation based on both visual and non-visual features such as haptics features. The multi-modal representation will allow us to infer visual information based on haptics information and vice-versa. The non-visual features are useful at object recognition in case of missing or incomplete visual observations. We have developed a 3D compositional part-based object representation, where the non-visual features are represented by tactile information after grasping an object.
\item[Relation with the deliverable] The technical report introduces a multi-modal representation which incorporates both visual and non-visual features.
\item[Attachment] (following pages until next annex) 
\end{description}
\includepdf[pages=-]{./attachedPapers/rezapouretal-Tech2015.pdf}

\subsection{Technical Report: 3D Compositional Models associated with Grasp Templates and Haptic Data} \label{ann:techReportAlex}
\begin{description}
\item[Authors] Alexander Rietzler.
\item[Info] Internal technical report.
\item[Abstract] The objective of this technical report is to associate graspability and haptic characteristics to object parts produced by a 3D compositional model. The advantage of the proposed method resides in the ability to perform robust grasping of novel object instances, composed of parts similar to the ones existing in the training set.
\item[Relation with the deliverable] The technical report introduces a method for associating grasp templates to object parts obtained using the 3D compositional model presented in Task 2.3.
\item[Attachment] (following pages) 
\end{description}
\includepdf[pages=-]{./attachedPapers/rietzeler-techReport.pdf}

\end{document}
