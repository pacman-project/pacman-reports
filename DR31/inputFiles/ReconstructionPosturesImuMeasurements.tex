%%% Low-cost, Fast and Accurate Reconstruction of Robotic and Human Postures
%%% via IMU Measurements

\subsubsection{Low-cost, Fast and Accurate Reconstruction of Robotic and Human Postures via IMU Measurements}
\label{sec:IMUGlove}

In this section, we describe our previously published approach to reconstruct the posture of kinematic structures which do not easily lend themselves to the use of rotary encoders. This part is motivated by our need to measure the joint angles of the Pisa/IIT SofHand --- its particular joint structure does not allow to fit encoders ---  after it has wrapped around an object (e.g., in an enveloping grasp), for the sake of employing it as a probe to explore and recognize objects.

The above discussed need, along with the mentioned constraints, brought us to consider the general problem of reconstructing the configurations of kinematic trees of rigid bodies without using measurements of relative angles, but employing absolute attitude sensors, such as IMUs, along with suitable filter algorithms. We argue that the relatively larger inaccuracies shown by absolute sensors can be compensated by suitable processing, such as passive complementary filters exploiting the Mahony-Hamel formulation. The proposed method is applicable to general kinematic structures where measurements of relative angles is not feasible or convenient, or where, as in the case of the Pisa/IIT SoftHand, the joint kinematics are not lower pairs. In the accompanying paper~\cite{Santaera:ICRA:2015} we present quantitative comparisons with ground truth data in grasping tests obtained for a two-fingered gripper: here, the fingers share the very same kinematic structure of the Pisa/IIT SoftHand. The comparisons validate the method employed, testifying that the resulting hardware design is mechanically robust, cheap and can be easily adapted to robotic hands with different topology, as well as to sensorizing gloves for studying human grasping strategies.

More details on this initial approach and on the achieved results can be found in the attached paper~\cite{Santaera:ICRA:2015} available at~\href{./attachedPapers/ReconstructionPosturesImuMeasurements.pdf}{this link}.  