%!TEX root = MS42.tex
%\section{Introduction}

In recent years, there has been growing interest in the development of a standardized framework for robotic implementations, where the robots interdisciplinary code and software could be easy to write and accessible to everyone. This is the philosophy of the \textit{Robot Operating System} (\textbf{ROS}), a flexible framework which provides all the instruments to develop and build robust and robot-independent applications, allowing researchers from different fields of robotics to share each others works with the great result of spreading the know-how.

Since ROS is the software framework used in PaCMan we developed a hardware interface for the robots used in the experimental platforms. However, this interface is just useful to command the robot using torque references. There are different strategies to generate the torque references to move a robot manipulator, they are designed for different situation for example under uncertainty, which main problem faced in PaCMan Project, in this case in the robot dynamics. Other strategies to deal with multiple goal references at kinematic and dynamic level were also implemented.

% \subsection{Project Goals}

% The aim of this milestone report is focused on the development of a ROS-enabled software environment for controlling the manipulator KUKA LWR IV, used in the PaCMan platforms. 

After a first phase of studying and analyzing some of the novel approaches for robot control, the default control strategies provided by the manufacturer has been extended, implementing a new set of controllers. 
The development of the controllers, including tuning and debugging, has been realized with the support of \textit{Gazebo}, a high performance realistic simulator which, interacting with ROS, gives the opportunity to test the efficiency of user-implemented control algorithms. 

Finally, the last phase consisted in pursuing experiments to test the implemented control strategies on the physical robot, arising a variety of practical issues which don't come out in simulation, that are explained in detail in this document.

\newpage
