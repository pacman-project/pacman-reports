%!TEX root = MS42.tex
\section{Introduction}
In recent years, there has been growing interest in the development of a standardized framework for robotic implementations, where the robots interdisciplinary code and software could be easy to write and accessible to everyone. This is the philosophy of the \textit{Robot Operating System} (\textbf{ROS}), a flexible framework which provides all the instruments to develop and build robust and robot-independent applications, allowing researchers from different fields of robotics to share each others works with the great result of spreading the know-how.

\subsection{Project Goals}
The aim of this project focused on the development of a ROS-enabled software environment for controlling the manipulator KUKA LWR IV. After a first phase of studying and analyzing some of the novel approaches of manipulators control, the default control strategies provided by the manufacturer has been extended, implementing a new set of controllers. The development of the controllers, including tuning and debugging, has been realized with the support of \textit{Gazebo}, a high performance realistic simulator which, interacting with ROS, gives the opportunity to test the efficiency of user-implemented control algorithms. Finally, the last phase consisted in the experimentation of the above mentioned control strategies on the physical robot, arising a variety of practical issues which don't come out in simulation, that are explained in details in the related section.

\newpage
