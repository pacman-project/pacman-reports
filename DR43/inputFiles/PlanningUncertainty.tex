%%% Learning and Inference of Dexterous Grasps for Novel Objects with Underactuated Hands

\subsubsection{Planning of Grasp Acquisition under Uncertainty and Incompleteness}
\label{sec:PlanningUncertainty}

In this work we have now ported our framework for reach to grasp planning under uncertainty to a real robot. We have also carried out extensive simulation analysis. The approach requires the robot to find collision-free trajectories that are robust in the face of such uncertainty. There are two fundamentally different approaches. First, a trajectory may be formulated that is robust to current uncertainty, but does not reason about how future information may reduce that uncertainty. Second, the robot may plan a trajectory to gather information that will reduce the uncertainty, so as to make a final grasping trajectory more reliable. Previous work typically separates these two aspects, separately planning information gathering trajectories and grasping trajectories. The two can be theoretically joined in a continuous state and action POMDP, but this leads to an infinite dimensional belief space planning problem that is hard to solve. We instead pursued a more tractable approach to combine infor- mation gathering and reach-to-grasp trajectories. Our main insight is we can embed the value of information in the much lower dimensional physical space to avoid the full complexity of belief state planning. This gives a well posed and tractable problem for reach-to-grasp planning under uncertainty. The specific contributions of our work are to: (i) plan information gain whilst simultaneously attempting to grasp the target object; (ii) encode expected information gain by warping distances in the workspace, creating a non-Euclidean metric that is information-sensitive; (iii) employ a hierarchical planning approach to reduce planning complexity in this space; (iv) update the belief about the object’s pose using a tactile observation model for a multi-finger hand palpating the object; (v) evaluate different methods, proving that our approach improves reach-to-grasp planning for a dexterous robot. We also use the grasp learning algorithm to generate the final target grasp. We assume a possibly incomplete shape model of the object is previously obtained from sensing. The work has been demonstrated in trials in simulation and on Boris. Empirical results confirm that sequential re-planning achieves a greater success rate than single grasp attempts, and that the information gain approach requires fewer iterations before a grasp is achieved. This is now written up into a journal paper for submission which is attached.

