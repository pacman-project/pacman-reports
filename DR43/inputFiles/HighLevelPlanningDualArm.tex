%%% High-Level Planning for Dual Arm Goal-Oriented Tasks

\subsubsection{On the Problem of Moving Objects with Autonomous Robots: a Unifying High-Level Planning Approach}
\label{sec:HighLevelPlanning}

In this work we propose a high-level planner that unifies
different object moving strategies such as pick-and-place and
handoff, and exploits support surfaces if required to achieve
the goal. Both a real and simulated examples with different
objects and multiple manipulators show the efficacy and scalability of this approach.

Our main contribution with this paper is a graph-based modelization of the problem, associated to a novel planning algorithm, that effectively unifies previously described strategies.
In this graph, each node represents the state of the object to be moved, while the arcs correspond to feasible change of state (more details on this are given in Sec. IV in the annex (Sec.~\ref{ann:highLevelPlanning})). In a hierarchical planning layered architecture, our planner needs to be placed below a symbolic reasoning planner.
The experimental validation of the proposed approach has been carried out on the Vito robot of Centro di Ricerca ``E. Piaggio''. In order to test the planner on a more complex setup with multiple end-effectors, a simulation with five Kuka LWR and one conveyor belt has been also performed.

More details on this new approach and on the achieved results can be found in Sec.~\ref{ann:highLevelPlanning}.
