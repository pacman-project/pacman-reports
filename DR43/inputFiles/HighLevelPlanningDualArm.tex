%%% High-Level Planning for Dual Arm Goal-Oriented Tasks

\subsubsection{High-Level Planning for Bimanual Object Passing}
\label{sec:HighLevelPlanningDualArm}

In order to successfully complete Task~5.3, which requires the object to be passed between the hands of the robot, we describe in this section our approach to solve high-level planning for this matter.

Our idea considers the possibility of having the object in a known position, which can be recognized using vision, and a higher-level agent (such as a user) decides a target configuration the object has to reach.

From this information, a semantic graph is constructed which has as nodes a grasp id and a workspace id, and as arcs all possible known transitions between nodes.

Once a minimum cost path has been found, it is translated back into Cartesian information for collision-free motion planning, grasp commands, and all low-level requirements to execute the plan successfully.

While there are many previous works on combining high-level semantic reasoning and low-level cartesian path planning (see e.g. \cite{karlsson2012combining, leidner2012things, leidner2013hybrid}) which mostly take advantage of object-specific reasoning introduced in \cite{levison1996connecting}, our main contribution is to include the possibility of passing an object between two end-effectors.

Specifically, considering end-effectors with different properties, the table (and possibly other non-movable surfaces) has its own set of grasps for an object, and is treated exactly like a hand when generating arcs in the graph, apart from the fact that it cannot be moved and, thus, there will be no arc connecting the same ``table grasp'' in adjacent workspaces (while there could be one if the grasp is performed via a hand of the robot).

Thus, considering that previous approaches mostly rely on a table in order to move an object from one arm workspace to another, we can simply classify the \emph{primitives} which allow a transition from a grasp/workspace pair into another in three categories:
\begin{enumerate}
	\item \emph{pick} with an end-effector from a fixed surface
	\item \emph{move} the end-effector
	\item \emph{place} with an end-effector onto a fixed surface
\end{enumerate}

Our approach uses instead a more general action of ``grasp transfer'', which can be specialized in:
\begin{enumerate}
	\item moving actions of an end-effector among its reachable workspaces
	\item pick and place actions if one end-effector involved is non-movable (such as a table)
	\item a grasp-ungrasp sequence if both end-effectors involved are movable
\end{enumerate}
This list is still under development, and we will try to generalize it even more in the future with other, more interesting \emph{primitives} which could possibly exploit dynamics, friction, gravity, and so on.

A more detailed view on this method and on the achieved results can be seen in Sec.~\ref{ann:dualArmPlanning}. % available at this~\href{./attachedPapers/HighLevelPlanningBimanualObjectPassing.pdf}{link}.
